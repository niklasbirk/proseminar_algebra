\begin{definition}
    Seien $u_1,\dots,u_n: \RR \to \RR$ differenzierbar und $a_{jk} \in \RR$ für $j,k = 1,\dots,n$.
    Dann heißt
    \begin{equation*}
        \begin{aligned}
            u'_1(t) &= a_{11} u_1(t) + \dots + a_{1n} u_n(t)\\
            u'_2(t) &= a_{21} u_1(t) + \dots + a_{2n} u_n(t)\\
            &\vdots\\
            u'_n(t) &= a_{n1} u_1(t) + \dots + a_{nn} u_n(t)
        \end{aligned}
    \end{equation*}
    ein \emph{homogenes lineares Differentialgleichungssystem} (DGLS) (1. Ordnung).\\
    Das obige System lässt sich auch kompakt in der Form
    \begin{equation}\tag{DGLS}\label{eq:dgls}
        \vec{u}'(t) = A \vec{u}(t)
    \end{equation}
    schreiben, wobei $\vec{u}(t) = \begin{pmatrix} u_1(t)\\ \vdots\\ u_n(t) \end{pmatrix}, \vec{u}'(t) = \begin{pmatrix} u'_1(t)\\ \vdots\\ u'_n(t) \end{pmatrix}$
    und $A \in \RR^{n \times n}$
\end{definition}

\begin{definition}
    Ein~\ref{eq:dgls} zusammen mit einer Anfangsbedingung
    \begin{equation*}
        \vec{u}_0 \coloneqq \vec{u}(t_0) = \begin{pmatrix} u_{1_0}\\ \vdots\\ u_{n_0} \end{pmatrix} \in \RR^n
    \end{equation*}
    an einer Stelle $t_0 \in \RR$ nennt man ein \emph{C\textc{auchy}-Problem} oder \emph{Anfangswertproblem}.
\end{definition}

\begin{lemma}[Produktregel]\label{thm:matrixexp-derivative}
    Sei $\vec{u}: \RR \to \RR^{n}$ eine differenzierbare vektorwertige Funktion und $A \in \RR^{n \times n}$ eine konstante Matrix.
    Dann gilt
    \begin{equation*}
        \dfdx{}{t} \left( \e^{t A} \vec{u}(t) \right) = \e^{t A} \dfdx{\vec{u}(t)}{t} + A \e^{t A} \vec{u}(t).
    \end{equation*}
\end{lemma}

\begin{proof}
    Das Ergebnis von $\e^{t A} \in \RR^{n \times n}$ ist eine Matrix.
    Demnach ist $\e^{t A} \vec{u}(t) \in \RR^{n \times 1}$ ein Matrix-Vektor-Produkt, deren Komponenten sich durch
    \begin{equation*}
        (\e^{t A} \vec{u}(t))_i = \sum^n_{j=1} \e^{t A}_{ij} \cdot u_j(t)
    \end{equation*}
    ergeben.
    Differenzieren der Komponenten, zusammen mit Linearität des Differentialoperators und der üblichen Produktregel, liefert
    \begin{align*}
        \dfdx{}{t}(\e^{t A} \vec{u}(t))_i
        &= \dfdx{}{t} \left( \sum^n_{j=1} \e^{t A}_{ij} \cdot u_j(t) \right)\\
        &= \dfdx{}{t} \left( \e^{t A}_{i1} \cdot u_j(t) + \dots + \e^{t A}_{in} \cdot u_j(t) \right)\\
        &=  \dfdx{}{t} \left( \e^{t A}_{i1} \cdot u_j(t) \right) + \dots + \dfdx{}{t} \left( \e^{t A}_{in} \cdot u_j(t) \right)\\
        &=  \left( \e^{t A}_{i1} \cdot \dfdx{u_j(t)}{t} + \dfdx{\e^{t A}_{i1}}{t} \cdot u_j(t) \right)
            + \dots + \left( \e^{t A}_{in} \cdot \dfdx{u_j(t)}{t} + \dfdx{\e^{t A}_{in}}{t} \cdot u_j(t) \right)\\
        &= \sum^n_{j=1} \left( \e^{t A}_{ij} \cdot \dfdx{u_j(t)}{t} + \dfdx{\e^{t A}_{ij}}{t} \cdot u_j(t) \right)\\
        &= \sum^n_{j=1} \left( \e^{t A}_{ij} \cdot \dfdx{u_j(t)}{t} \right)
            +  \sum^n_{j=1} \left( \dfdx{\e^{t A}_{ij}}{t} \cdot u_j(t) \right).
    \end{align*}
    Das ist ganzheitlich betrachtet gerade
    \begin{equation*}
        \dfdx{}{t} \left( \e^{t A} \vec{u}(t) \right) = \e^{t A} \dfdx{\vec{u}(t)}{t} + \dfdx{\e^{t A}}{t} \vec{u}(t)
        = \e^{t A} \dfdx{\vec{u}(t)}{t} + A \e^{t A} \vec{u}(t).
    \end{equation*}
\end{proof}

\begin{remark*}
    Die Produktregel gilt allgemein auch für das Differenzieren von Matrizenprodukten.
    Seien $M$ und $N$ Matrizen, deren Komponenten Funktionen von $t$ sind.
    Dann gilt
    \begin{equation*}
        \dfdx{}{t}(M(t) N(t)) = M(t) \dfdx{N(t)}{t} + \dfdx{M(t)}{t} N(t).
    \end{equation*}
\end{remark*}

\begin{corollary}\label{thm:matrixexp-derivative-corollary}
    Sei $\vec{u}_0 \in \RR^n$ konstant und $A \in \RR^{n \times n}$.
    Dann gilt
    \begin{equation*}
        \dfdx{}{t} \left( \e^{(t - t_0) A} \vec{u}_0 \right) = A \e^{(t - t_0) A} \vec{u}_0.
    \end{equation*}
\end{corollary}

\begin{theorem}[Existenz und Eindeutigkeit]\label{thm:existenz-eindeutigkeit}
    Vorgelegt sei ein C\textc{auchy}-Problem
    \begin{equation}\tag{CP}\label{eq:cp}
        \vec{u}'(t) = A \vec{u}(t), \qquad \vec{u}_0 \coloneqq \vec{u}(t_0) = \begin{pmatrix} u_{1_0}\\ \vdots\\ u_{n_0} \end{pmatrix}.
    \end{equation}
    Dann besitzt~\eqref{eq:cp} eine eindeutig bestimmte Lösung $\vec{u}$ auf $\RR$ mit der Form
    \begin{equation}\tag{$\ast$}\label{eq:solution}
        \vec{u}(t) = \e^{(t - t_0) A} \vec{u}_0.
    \end{equation}
\end{theorem}

\begin{proof}
    \begin{itemize}
        \item   \underline{Existenz:}\\
                Einsetzen von~\eqref{eq:solution} in die rechte Seite von~\eqref{eq:cp} liefert
                \begin{equation*}
                    A \vec{u}(t) = A \e^{(t - t_0) A} \vec{u}_0.
                \end{equation*}
                Zusammen mit~\hyperref[thm:matrixexp-derivative]{Lemma~\ref*{thm:matrixexp-derivative}}
                bzw.~\hyperref[thm:matrixexp-derivative-corollary]{Folgerung~\ref*{thm:matrixexp-derivative-corollary}}
                folgt direkt, dass~\eqref{eq:solution} das C\textc{auchy}-Problem löst.

        \item   \underline{Eindeutigkeit:}\\
                Angenommen $\vec{v}(t)$ sei eine weitere Lösung, d.h.~es gilt $\vec{v}' = A \vec{v},\ \vec{v}(t_0) = \vec{u}_0$.
                Dann ist
                \begin{align*}
                    \dfdx{}{t} \left( \e^{-(t - t_0) A} \vec{v}(t) \right)
                    \overset{\ref{thm:matrixexp-derivative}}&{=} \e^{-(t - t_0) A} \dfdx{\vec{v}(t)}{t} - A \e^{-(t - t_0) A} \vec{v}(t)\\
                    &= \e^{-(t - t_0) A} A \vec{v}(t) - \e^{-(t - t_0) A} A \vec{v}(t)\\
                    &= \vec{0}. % TODO: e^{-tA} kommutiert mit A, weil A mit sich selbst kommutiert; Minh? oder selsbt sagen?
                \end{align*}
                Da die Ableitung die Nullmatrix ist, ist $\e^{-(t - t_0) A} \vec{v}(t)$ konstant, d.h.
                \begin{equation*}
                    \e^{-(t - t_0) A} \vec{v}(t) = \vec{c}.
                \end{equation*}
                Die Inverse der Matrixexponentialfunktion existiert immer, weshalb
                \begin{equation*}
                   \vec{v}(t) =  \e^{(t - t_0) A} \vec{c}
                \end{equation*}
                folgt.
                Mit Voraussetzung $\vec{v}(t_0) = \vec{u}_0$ gilt dann für die Konstante $\vec{c}$
                \begin{equation*}
                    \vec{v}(t_0) =  \e^{(t_0 - t_0) A} \vec{c} = E \vec{c} = \vec{c} = \vec{u}_0.
                \end{equation*}
                Insgesamt ergibt sich also
                \begin{equation*}
                    \vec{v}(t) =  \e^{(t - t_0) A} \vec{u}_0 = \vec{u}(t)
                \end{equation*}
    \end{itemize}
\end{proof}