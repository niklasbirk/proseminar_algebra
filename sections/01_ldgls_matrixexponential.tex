\begin{definition}
    Seien $y_1,\dots,y_n: \RR \to \RR$ differenzierbar und $a_{jk} \in \RR$ für $j,k = 1,\dots,n$.
    Dann heißt
    \begin{equation*}
        \begin{aligned}
            y'_1(t) &= a_{11} y_1(t) + \dots + a_{1n} y_n(t)\\
            y'_2(t) &= a_{21} y_1(t) + \dots + a_{2n} y_n(t)\\
            &\vdots\\
            y'_n(t) &= a_{n1} y_1(t) + \dots + a_{nn} y_n(t)
        \end{aligned}
    \end{equation*}
    ein \emph{homogenes lineares Differentialgleichungssystem} (DGLS) (1. Ordnung).\\
    Das obige System lässt sich auch kompakt in der Form
    \begin{equation}\tag{DGLS}\label{eq:dgls}
        \vec{y}'(t) = A \vec{y}(t)
    \end{equation}
    schreiben, wobei $\vec{y}(t) = \begin{pmatrix} y_1(t)\\ \vdots\\ y_n(t) \end{pmatrix}, \vec{y}'(t) = \begin{pmatrix} y'_1(t)\\ \vdots\\ y'_n(t) \end{pmatrix}$
    und $A \in \RR^{n \times n}$
\end{definition}

\begin{definition}
    Ein~\ref{eq:dgls} zusammen mit einer Anfangsbedingung
    \begin{equation*}
        \vec{y}(t_0) = \vec{y}_0 \coloneqq \begin{pmatrix} y_{1_0}\\ \vdots\\ y_{n_0} \end{pmatrix} \in \RR^n
    \end{equation*}
    an einer Stelle $t_0 \in \RR$ nennt man ein \emph{C\textc{auchy}-Problem} oder \emph{Anfangswertproblem}.
\end{definition}

\begin{theorem}[Existenz und Eindeutigkeit]\label{thm:existenz-eindeutigkeit}
    Vorgelegt sei ein C\textc{auchy}-Problem
    \begin{equation}\tag{CP}\label{eq:cp}
        \vec{y}'(t) = A \vec{y}(t), \qquad \vec{y}(t_0) = \vec{y}_0 \coloneqq \begin{pmatrix} y_{1_0}\\ \vdots\\ y_{n_0} \end{pmatrix}.
    \end{equation}
    Dann besitzt~\eqref{eq:cp} eine eindeutig bestimmte Lösung $\vec{y}$ auf $\RR$ mit der Form
    \begin{equation}\tag{$\ast$}\label{eq:solution}
        \vec{y}(t) = e^{(t - t_0) A} \vec{y}_0.
    \end{equation}
\end{theorem}

\begin{proof}
    \begin{itemize}
        \item   \underline{Existenz:}\\
                Einsetzen von~\eqref{eq:solution} in die rechte Seite von~\eqref{eq:cp} liefert
                \begin{equation*}
                    A \vec{y}(t) = A e^{(t - t_0) A} \vec{y}_0.
                \end{equation*}
                Zusammen mit~\eqref{*} folgt direkt, dass~\eqref{eq:solution} das C\textc{auchy}-Problem löst.

        \item   \underline{Eindeutigkeit:}\\
                Angenommen $\vec{u}(t)$ sei eine weitere Lösung, d.h.~es gilt $\vec{u}' = A \vec{u},\ \vec{u}(t_0) = \vec{y}_0$.
                Dann ist
                \begin{align*}
                    \dfdx{}{x} \left( \text{tbc.} \right)
                \end{align*}
    \end{itemize}
\end{proof}