Ein~\ref{eq:dgls} zusammen mit einem einem Anfangswertvektor
\begin{equation*}
    \vec{y}(x_0) \coloneqq \vec{y}_0 \coloneqq \begin{pmatrix} y_{1_0}\\ \vdots\\ y_{n_0} \end{pmatrix} \in \RR^n
\end{equation*}
an der Stelle $x_0 \in \RR$ nennt man ein \emph{C\textc{auchy}-Problem} oder \emph{Anfangswertproblem}.

\begin{theorem}[Existenz- und Eindeutigkeit]
    Vorgelegt sei ein C\textc{auchy}-Problem
    \begin{equation}\tag{CP}\label{eq:cp}
        \vec{y}'(x) = A \vec{y}(x), \qquad \vec{y}_0 \coloneqq \begin{pmatrix} y_{1_0}\\ \vdots\\ y_{n_0} \end{pmatrix}.
    \end{equation}
    Dann besitzt~\eqref{eq:cp} eine eindeutig bestimmte Lösung $\vec{y}$ auf $\RR$ mit der Form
    \begin{equation}\tag{$\ast$}\label{eq:solution}
        \vec{y}(x) = e^{(x - x_0) A} \vec{y}_0.
    \end{equation}
\end{theorem}

\begin{proof}
    \begin{itemize}
        \item   \underline{Existenz:}\\
                Einsetzen in die rechte Seite von~\eqref{eq:solution} in~\eqref{eq:cp} liefert
                \begin{equation*}
                    A \vec{y}(x) = A e^{(x - x_0) A} \vec{y}_0.
                \end{equation*}
                Zusammen mit~\eqref{*} folgt direkt, dass~\eqref{eq:solution} das C\textc{auchy}-Problem löst.

        \item   \underline{Eindeutigkeit:}\\
                Angenommen $\vec{u}(x)$ sei eine weitere Lösung, d.h.~es gilt $\vec{u}' = A \vec{u},\ \vec{u}(x_0) = \vec{y}_0$.
                Dann ist
                \begin{align*}
                    \dfdx{}{x} \left( \text{tbc.} \right)
                \end{align*}
    \end{itemize}
\end{proof}