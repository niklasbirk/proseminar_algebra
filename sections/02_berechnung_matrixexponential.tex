Wie in~\hyperref[thm:existenz-eindeutigkeit]{Satz~\ref*{thm:existenz-eindeutigkeit}} festgestellt wurde,
kann die Lösung eines~\hyperref[eq:cp]{C\textc{auchy}-Problems} mit der Matrixexponentialfunktion berechnet werden.
In diesem Abschnitt soll nun herausgearbeitet werden, wie sich die Matrixexponentialfunktion selbst berechnen lässt.

\subsection{Diagonale Matrizen}\label{subsec:diagonale-matrizen}
Zunächst werden diagonale Matrizen
\begin{equation*}
    A = \diag\{a_1,\dots,a_n\}
    = \begin{pmatrix}
          a_1 &        & \\
              & \ddots & \\
              &        & a_3
    \end{pmatrix}
\end{equation*}
betrachtet.

\begin{theorem}\label{thm:matrixexponential-diagonal}
    Sei eine Diagonalmatrix $A = \diag\{a_1,\dots,a_n\}$.
    Dann gilt für die Matrixexponentialfunktion
    \begin{equation*}
        \e^{t A} = \diag\{\e^{t a_1},\dots,\e^{t a_n}\}.
    \end{equation*}
\end{theorem}

\begin{proof}
    Es gilt
    \begin{equation*}
        A^k = \begin{pmatrix}
                  a^k_1 &        & \\
                  & \ddots & \\
                  &        & a^k_n
        \end{pmatrix}.
    \end{equation*}
    Womit für die Matrixexponentialfunktion folgt, dass
    \begin{alignat*}{2}
        \e^{t A}
        &= \sum^\infty_{k=0} \frac{(tA)^k}{k!}
            &&= \sum^\infty_{k=0} \frac{t^k A^k}{k!}\\
        &= \sum^\infty_{k=0} \frac{t^k}{k!}
                \begin{pmatrix}
                    a^k_1 &        & \\
                          & \ddots & \\
                          &        & a^k_n
                \end{pmatrix}
            &&= \sum^\infty_{k=0}
                \begin{pmatrix}
                    \frac{t^k a^k_1}{k!} &        & \\
                                         & \ddots & \\
                                         &        & \frac{t^k a^k_n}{k!}
                \end{pmatrix}\\
        &= \begin{pmatrix}
                    \sum\limits^\infty_{k=0} \frac{(t a_1)^k}{k!} &        & \\
                                                                 & \ddots & \\
                                                                 &        & \sum\limits^\infty_{k=0} \frac{(t a_n)^k}{k!}
                \end{pmatrix}
            &&= \begin{pmatrix}
                   \e^{t a_1} &        & \\
                              & \ddots & \\
                              &        & \e^{t a_n}
            \end{pmatrix}
    \end{alignat*}
\end{proof}


\subsection{Diagonalisierbare Matrizen}\label{subsec:diagonalisierbare-matrizen}
Das Berechnen einer Matrixexponentialfunktion mit einer diagonalisierbaren Matrix $A$ im Exponenten
kann auf den obigen Fall der~\hyperref[subsec:diagonale-matrizen]{diagonalen Matrizen} zurückgeführt werden.

\begin{definition}[Wiederholung]
    Eine Matrix $A$ heißt \emph{diagonalisierbar}, falls eine Diagonalmatrix $D$ und eine reguläre Matrix $A$ so existieren, dass
    \begin{equation*}
        A = S D S^{-1}
    \end{equation*}
    gilt.
\end{definition}

\begin{lemma}\label{thm:matrixexponential-diagonalisierbar}
    Sei $A$ ähnlich zu $B$ mit $A = S B S^{-1}$.
    Dann gilt für die Matrixexponentialfunktion
    \begin{equation*}
        \e^{t A} = S \e^{t B} S^{-1}.
    \end{equation*}
\end{lemma}

\begin{proof}
    Zuerst wird gezeigt, dass $A^k = S B^k S^{-1}$ für $k \in \NN$:
    \begin{alignat*}{2}
        A^k &= (S B S^{-1})^k &&= \underbrace{(S B S^{-1}) (S B S^{-1}) (S B S^{-1}) \dots (S B S^{-1})}_{k \text{-mal}}\\
            &= SB\underbrace{(S^{-1} S)}_{E} B \underbrace{(S^{-1} S)}_{E} \dots \underbrace{(S^{-1} S)}_{E} B S^{-1}
                &&= V \underbrace{L \cdots L}_{k \text{-mal}} S^{-1}\\
            &= S B^k S^{-1}
    \end{alignat*}
    Damit folgt dann für die Matrixexponentialfunktion
    \begin{alignat*}{2}
        \e^{t A} &= \e^{t \cdot S B S^{-1}} &&= \sum^\infty_{k=0} \frac{(t \cdot S B S^{-1})^k}{k!}\\
                 &= \sum^\infty_{k=0} \frac{t^k (S B S^{-1})^k}{k!} &&= \sum^\infty_{k=0} \frac{t^k \cdot S B^k S^{-1}}{k!}\\
                 &= V \left( \sum^\infty_{k=0} \frac{t^k B^k}{k!} \right) S^{-1} &&= V \left( \sum^\infty_{k=0} \frac{(t B)^k}{k!} \right) S^{-1}\\
                 &= V \e^{t B} S^{-1}
    \end{alignat*}
\end{proof}

\begin{corollary}
    Sei $A$ diagonalisierbar mit $A = S D S^{-1}$.
    Dann gilt für die Matrixexponentialfunktion
    \begin{equation*}
        \e^{t A} = S \e^{t D} S^{-1}.
    \end{equation*}
\end{corollary}

\begin{proof}
    $A$ ist nach Definition genau dann diagonalisierbar, wenn $A$ zu einer Diagonalmatrix $D$ ähnlich ist.
    Also~\hyperref[thm:matrixexponential-diagonalisierbar]{Lemma~\ref*{thm:matrixexponential-diagonalisierbar}}.
\end{proof}

\subsection{Nicht-diagonalisierbare Matrizen}\label{subsec:nichtdiagonalisierbare-matrizen}
Jede quadratische Matrix $A$ lässt sich auf eine \emph{J\textc{ordan}-Normalform} $J = V^{-1} A V$ bringen.
Eine J\textc{ordan}-Normalform $J$ ist eine Blockdiagonalmatrix der Form
\begin{gather*}
    J = \begin{pmatrix}
            J_1 &        &  \\
                & \ddots & \\
                &        & J_n
    \end{pmatrix}\\
    \intertext{mit J\textc{ordan}-Block}
    J_i = \begin{pmatrix}
              \lambda_i & 1         &        & \\
                        & \lambda_i & \ddots & \\
                        &           & \ddots & 1 \\
                        &           &        & \lambda_i
    \end{pmatrix}
    \qquad (i = 1,\dots,n).
\end{gather*}
Das folgende \hyperref[thm:blockdiag-exp]{Lemma~\ref*{thm:blockdiag-exp}} lässt vermuten,
dass die gewonnenen Erkenntnisse aus~\autoref{subsec:diagonalisierbare-matrizen} zur Berechnung der
Matrixexponentialfunktion $\e^{A} = \e^{V J V^{-1}}$ anwendbar sein könnten.

\begin{lemma}\label{thm:blockdiag-exp}
    Sei $A \in \RR^{n \times n}$ eine Blockdiagonalmatrix
    \begin{equation*}
        A = \diag\left\{A_1,\dots,A_m\right\}
        = \begin{pmatrix}
                A_1 &        & \\
                    & \ddots & \\
                    &        & A_m
        \end{pmatrix},
    \end{equation*}
    mit quadratischen Matrizen $A_1, \dots, A_m$.
    Dann gilt
    \begin{equation*}
        \e^{t A} = \diag\left\{ \e^{t A_1}, \dots,\e^{t A_m} \right\}.
    \end{equation*}
\end{lemma}

\begin{proof}
    Beweis analog~\hyperref[thm:matrixexponential-diagonal]{Satz~\ref*{thm:matrixexponential-diagonal}}
\end{proof}

Zudem zeigt~\hyperref[thm:blockdiag-exp]{Lemma~\ref*{thm:blockdiag-exp}},
dass sich die Berechnung von $\e^J$ auf die einzelnen $\e^{J_i}$ für $i = 1,\dots,n$ reduzieren lässt.
Dafür werden im Folgenden die J\textc{ordan}-Blöcke $J_i$ betrachtet,
denn ein J\textc{ordan}-Block $J_i$ lässt sich weiter zerlegen in
\begin{equation*}
    J_i = \begin{pmatrix}
              \lambda_i & 1         &        & \\
                        & \lambda_i & \ddots & \\
                        &           & \ddots & 1 \\
                        &           &        & \lambda_i
    \end{pmatrix}
    = \underbrace{\begin{pmatrix}
        \lambda_i &        & \\
                  & \ddots & \\
                  &        & \lambda_i
    \end{pmatrix}}_{= L_i}
    + \underbrace{\begin{pmatrix}
          0 & 1 &        & \\
            & 0 & \ddots & \\
            &   & \ddots & 1 \\
            &   &        & 0
    \end{pmatrix}}_{\eqqcolon N}
    = L_i + N.
\end{equation*}

\begin{definition}[Wiederholung]
    Eine Matrix $N$ heißt \emph{nilpotent}, falls ein $l \in \NN$ so existiert, dass $N^l = 0$.
\end{definition}

\begin{corollary}
    Die Matrix $N \in \NN^{l \times l}$ aus der Zerlegung $J_i = L_i + N$ ist eine nilpotente Matrix mit $N^l = 0$.
\end{corollary}

Betrachtet man den Vorgang des Potenzierens von $N$ ausführlicher, erkennt man,
dass die Nebendiagonalen auf denen die $1$en stehen \enquote{nach oben-rechts hinaus} geschoben werden:
\begin{equation*}
    N = \begin{pmatrix}
             0 & 1      &        & \\
               & \ddots & \ddots & \\
               &        & \ddots & 1\\
               &        &        & 0
         \end{pmatrix},
    N^2 = \begin{pmatrix}
               0 & 0      & 1      &        & \\
                 & \ddots & \ddots & \ddots & \\
                 &        & \ddots & \ddots & 1\\
                 &        &        & \ddots & 0\\
                 &        &        &        & 0
           \end{pmatrix},
    \dots,
    N^{l-1} = \begin{pmatrix}
                   0 &        & 1\\
                     & \ddots & \\
                     &        & 0
              \end{pmatrix},
    N^l = \vec{0}_l
\end{equation*}

\begin{lemma}
    Die Matrizen $L_i$ und $N$ aus der Zerlegung $J_i = L_i + N$ kommutieren bzgl.\ der Matrixmultiplikation.
\end{lemma}

\begin{proof}
    \begin{equation*}
        L_i N = (\lambda_i E) N = \lambda_i (EN) = \lambda_i (NE) = N (\lambda_i E) = N L_i,
    \end{equation*}
\end{proof}

Damit gilt
\begin{equation*}
    \e^{t (L_i + N)} = \e^{t L_i + t N} = \e^{t L_i} \e^{t N}.
\end{equation*}

Da $L_i$ eine Diagonalmatrix ist, ist die Lösung von $\e^{t L_i}$ bereits aus~\autoref{subsec:diagonale-matrizen} bekannt.
Für $N$ als nilpotente Matrix ergibt sich zudem eine endliche Summe bei der Berechnung von $\e^{t N}$,
da ab einem $l \in \NN$ gilt, dass $N^l = N^{l+1} = \dots = 0$, d.h.
\begin{align*}
    \e^ {tN } &= \sum^\infty_{k=0} \frac{(t N)^k}{k!}
        = \sum^{l-1}_{k=0} \frac{(t N)^k}{k!}
        = E + t N + \frac{t^2}{2} N^2 + \dots + \frac{t^{l-1}}{(l-1)!} N^{l-1}\\
    &= E + t N +  \frac{t^2}{2} \begin{pmatrix}
                                0 & 0      & 1      &        & \\
                                  & \ddots & \ddots & \ddots & \\
                                  &        & \ddots & \ddots & 1\\
                                  &        &        & \ddots & 0\\
                                  &        &        &        & 0
                            \end{pmatrix}
        + \dots + \frac{t^{l-1}}{(l-1)!}  \begin{pmatrix}
                                        0 &        & 1\\
                                          & \ddots & \\
                                          &        & 0
                                    \end{pmatrix}\\
    &= \begin{pmatrix}
           1 & t & \frac{t^2}{2} & \frac{t^3}{3!} & \dots  & \frac{t^{l-1}}{(l-1)!}\\
           & 1 & t               & \frac{t^2}{2}  & \dots  & \frac{t^{l-2}}{(l-2)!}\\
           &   & 1               & t              & \ddots  & \vdots\\
           &   &                 & 1              & \ddots & \frac{t^2}{2}\\
           &   &                 &                & \ddots & t\\
           &   &                 &                &        & 1
    \end{pmatrix}.
\end{align*}

\begin{remark*}
    Für die Lösung eines~\hyperref[eq:cp]{C\textc{auchy}-Problems} mit einer nicht-diagonalisierbaren Matrix $A = V J V^{-1}$
    mit J\textc{ordan}-Normalform $J$, gilt also insgesamt
    \begin{align*}
        \vec{u}(t) &= \e^{(t - t_0) A} \vec{u}_0 = \e^{(t - t_0) V L V^{-1}} \vec{u}_0 = V \e^{(t - t_0) J} V^{-1} \vec{u}_0\\
        &= V \begin{pmatrix}
                \e^{(t - t_0) J_1} &        & \\
                                   & \ddots & \\
                                   &        & \e^{(t - t_0) J_n}
            \end{pmatrix} V^{-1} \vec{u}_0\\
        &= V \begin{pmatrix}
                \e^{(t - t_0) L_1} \e^{(t - t_0) N} &        & \\
                                                    & \ddots & \\
                                                    &        & \e^{(t - t_0) L_n} \e^{(t - t_0) N}
            \end{pmatrix} V^{-1} \vec{u}_0.
    \end{align*}
\end{remark*}

\begin{example*}
    Gegeben sei das~\ref{eq:cp}
    \begin{equation*}
        \vec{u}'(t) = \underbrace{\begin{pmatrix} a & 1\\ 0 & a \end{pmatrix}}_{= A} y(t),\ a \in \RR,
        \qquad \vec{u}(t_0 = 0) = \vec{u}_0 = \begin{pmatrix}y_1\\ y_2\end{pmatrix}.
    \end{equation*}
    $A$ lässt sich zerlegen in eine Diagonalmatrix $D$ und nilpotente Matrix $N$:
    \begin{equation*}
        A = \underbrace{\begin{pmatrix} a & 0\\ 0 & a \end{pmatrix}}_{\eqqcolon D}
            + \underbrace{\begin{pmatrix} 0 & 1\\ 0 & 0 \end{pmatrix}}_{\eqqcolon N}
    \end{equation*}
    Nach~\hyperref[thm:existenz-eindeutigkeit]{Satz~\ref*{thm:existenz-eindeutigkeit}} und~\autoref{sec:berechnung-matrixexponential} ist die Lösung gegeben durch
    \begin{alignat*}{2}
        \vec{u}(t) &= \e^{t A} \vec{u}_0
            &&= \e^{t (D + N)} \vec{u}_0\\
        &= \e^{t D + t N} \vec{u}_0
            &&= \e^{t D} \e^{t N} \vec{u}_0\\
        &= \begin{pmatrix} \e^{t a} & 0\\ 0 & \e^{t a} \end{pmatrix} \begin{pmatrix} 1 & t\\ 0 & 1 \end{pmatrix} \vec{u}_0
            &&= \begin{pmatrix} \e^{t a} & t \e^{t a}\\ 0 & \e^{t a} \end{pmatrix} \begin{pmatrix}y_1\\ y_2\end{pmatrix}\\
        &= \begin{pmatrix} y_1 \e^{t a} + y_2 t \e^{t a}\\ y_2 \e^{t a} \end{pmatrix}
            &&= \begin{pmatrix} \e^{t a} (y_1 + y_2 t)\\ y_2 \e^{t a} \end{pmatrix}
    \end{alignat*}
\end{example*}